% <- percent signs are used to comment
\documentclass[12pt]{article}

%%%%%% PACKAGES - this part loads additional material for LaTeX %%%%%%%%%
% Nearly anything you want can be done in LaTeX if you load the right package 
% (search ctan.org or google it if you are looking for something).  We will load
% here a few that we need for this document or that we expect you to need later.

% The next 3 lines are needed to fix shortcomings of TeX that only make sense given its 40-year history ...
% Simple keep and ignore.
\usepackage[utf8]{inputenc}
\usepackage[T1]{fontenc}
\usepackage{lmodern}
\usepackage{amsmath}
\usepackage{changepage}
\usepackage{lipsum}

% Custom margins (and paper sizes etc.) because LaTeX else wastes much space
\usepackage[margin=1in]{geometry}

% The following packages are created by the American Mathematical Society (AMS)
% and provide lots of tools for special fonts, symbols, theorems, and proof
\usepackage{amsmath,amsfonts,amssymb,amsthm}
% mathtools contains many detail improvements over ams and core tex
\usepackage{mathtools}

% graphicx is required for images
\usepackage{graphicx}

% enumitem used for customizing enumerations
\usepackage[shortlabels]{enumitem}

% tikz is the package used for drawing, in particular for drawing trees. You may also find simplified packages like tikz-qtree and forest useful
\usepackage{tikz}

% hyperref allows links, urls, and many other PDF tricks.  We load it here
%          in such a way that the PDF file has info about it
\usepackage[%
	pdftitle={CS240 Assignment 0},%
	hidelinks,%
]{hyperref}


%%%%%% COMMANDS - here you can define your own LaTeX-commands %%%%%%%%%

%%%%%% End of Preamble %%%%%%%%%%%%%

\begin{document}

\begin{center}
{\Large\textbf{CS240, Spring 2022}}\\
\vspace{2mm}
{\Large\textbf{Assignment 3: Question 2}}\\
\vspace{3mm}
\end{center}
\[ \]
\begin{adjustwidth}{0em}{0pt}
\textbf{Q2a)} The best case will occur when $random.size(A.size) = 0$. This will result in only a constant number of operations will run for each call of ArrayAlg. The array reduces by 1, which means that we will have a runtime of:
\begin{align*}
    \begin{aligned}
       T(n) &= \sum^{n}_{i=0}1 \\
       T(n) &= n \\
       T(n) &\in O(n)  \\
    \end{aligned}
\end{align*}
\textbf{Q2b)} The worst case will occur when $random.size(A.size) = n-1$. This will result in only a loop which will have $n-1$ of operations will run for each call of ArrayAlg. The array reduces by 1, which means that we will have a runtime of:
\begin{align*}
    \begin{aligned}
       T(n) &= \sum^{n}_{i=0}(n-1) \\
       T(n) &= \frac{(n+1)(n)}{2} -n \\
       T(n) &= \frac{n^2}{2} - \frac{n}{2} \\
       T(n) &= O(n^2)
    \end{aligned}
\end{align*}
\textbf{Q2c)} For each iteration of ArrayAlg, we have a 2 possibilities: Either $random.size(A.size) = 0$ and so we will only do 1 operation, otherwise if $random.size(A.size) = x$ we will do $x$ operations as we will print x elements. Therefore our good and bad case looks like:
\begin{align*}
    T(A[0,...,A.Size]) = \begin{cases}
       1 + T(A[0,...,A.Size-2] \text{ where $random.size(A.size) = 0$} \\
       x + T(A[0,...,A.Size-2] \text{ where $random.size(A.size) = x$})
    \end{cases}
\end{align*}
Note that for each iteration of ArrayAlg, the "good case" will only happen once, and the base case will happen the remaining times. Thus:
\begin{align*}
    \begin{aligned}
       \text{number of good outcomes} = 1 \ \ & \ \ \text{number of bad outcomes} = n-1
    \end{aligned}
\end{align*}
It thus follows that:
\begin{align*}
    \begin{aligned}
       T^{exp}(n) &= \frac{1}{\prod_n} ( \sum_{\pi \in \prod_n:\pi good} T(\pi) + \sum_{\pi \in \prod_n:\pi bad} T(\pi) ) \\
       T^{exp}(n) &= \frac{1}{\prod_n} ( \sum_{\pi \in \prod_n:\pi good} 1 + T(n-2) + \sum_{\pi \in \prod_n:\pi bad} x + T(n-2) ) \\
       T^{exp}(n) &= \frac{1}{n} ( 1\times(1 + T(n-2)) + n\times(x + T(n-2)) )
    \end{aligned}
\end{align*}
Howver since $1 <= x <= n-1$ it follows that $x < n$ thus we get:
\begin{align*}
    \begin{aligned}
       T^{exp}(n) &= \frac{1}{n} ( 1\times(1 + T(n-1)) + (n-1)\times(x + T(n-1)) ) \\
       T^{exp}(n) &< \frac{1}{n} ( 1\times(1 + T(n-1)) + (n-1)\times(n + T(n-1)) ) \\
       T^{exp}(n) &< \frac{1}{n} ( 1 + T(n-1) + n^2 + nT(n-1) - n - T(n-1)) \\
       T^{exp}(n) &< \frac{1}{n} ( 1 + n^2 + nT(n-1) - n ) \\
       T^{exp}(n) &< \frac{1}{n} + n + T(n-1) - 1 \\
    \end{aligned}
\end{align*}
So our function will run $n$ times and each time it will do $n+\frac{1}{n} -1 $ operations and since:
\[ n ( n+\frac{1}{n} -1) \in O(n^2) \]
Therefore $ T^{exp}(n) \in O(n^2) $
\end{adjustwidth} 




\end{document}