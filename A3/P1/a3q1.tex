% <- percent signs are used to comment
\documentclass[12pt]{article}

%%%%%% PACKAGES - this part loads additional material for LaTeX %%%%%%%%%
% Nearly anything you want can be done in LaTeX if you load the right package 
% (search ctan.org or google it if you are looking for something).  We will load
% here a few that we need for this document or that we expect you to need later.

% The next 3 lines are needed to fix shortcomings of TeX that only make sense given its 40-year history ...
% Simple keep and ignore.
\usepackage[utf8]{inputenc}
\usepackage[T1]{fontenc}
\usepackage{lmodern}
\usepackage{amsmath}
\usepackage{changepage}
\usepackage{lipsum}

% Custom margins (and paper sizes etc.) because LaTeX else wastes much space
\usepackage[margin=1in]{geometry}

% The following packages are created by the American Mathematical Society (AMS)
% and provide lots of tools for special fonts, symbols, theorems, and proof
\usepackage{amsmath,amsfonts,amssymb,amsthm}
% mathtools contains many detail improvements over ams and core tex
\usepackage{mathtools}

% graphicx is required for images
\usepackage{graphicx}

% enumitem used for customizing enumerations
\usepackage[shortlabels]{enumitem}

% tikz is the package used for drawing, in particular for drawing trees. You may also find simplified packages like tikz-qtree and forest useful
\usepackage{tikz}

% hyperref allows links, urls, and many other PDF tricks.  We load it here
%          in such a way that the PDF file has info about it
\usepackage[%
	pdftitle={CS240 Assignment 0},%
	hidelinks,%
]{hyperref}


%%%%%% COMMANDS - here you can define your own LaTeX-commands %%%%%%%%%

%%%%%% End of Preamble %%%%%%%%%%%%%

\begin{document}

\begin{center}
{\Large\textbf{CS240, Spring 2022}}\\
\vspace{2mm}
{\Large\textbf{Assignment 2: Question 1}}\\
\vspace{3mm}
\end{center}
\[ \]
\begin{adjustwidth}{0em}{0pt}
\textbf{Q1)} From the lectures we are told the are equation for the average-case run-time is:
\[ T^{avg}(n) = \frac{\sum_{I:size(I)=n}T(I)}{\text{(number of instances with size n)}}\]
However, we know that each arrangement of the array will be equally likely. Since the array has a total of $n!$ permutations our equation for the average run-time becomes:
\[ T^{avg}(n) = \frac{1}{n!}\sum_{n \in \prod}T(\pi)\]
Each time we run through the array we can go through $i$ elements before we terminate, therefore out of $n$ elements we need to choose $i$ and the rest can be in any order. Thus our equation becomes:
\begin{align*}
    \begin{aligned}
       T^{avg}(n) &= \frac{1}{n!}\sum^{n}_{i = 0} {n \choose i} (n-i)! \\
       T^{avg}(n) &= \frac{1}{n!}\sum^{n}_{i = 0} \frac{n!}{i!} \\
       T^{avg}(n) &= \sum^{n}_{i = 0} \frac{1}{i!} \\
    \end{aligned}
\end{align*}
Note that the Taylor series expansion of $e^x$ is given by:
\begin{align*}
    \begin{aligned}
       e^x &= \sum^{\infty}_{i=0} \frac{x^i}{i!} \\
       e^1 &= \sum^{\infty}_{i=0} \frac{1^i}{i!} \\
       e &= \sum^{\infty}_{i=0} \frac{1}{i!} 
    \end{aligned}
\end{align*}
Thus it follows that:
\begin{align*}
    \begin{aligned}
       \sum^{n}_{i = 0} \frac{1}{i!} &\leq \sum^{\infty}_{i=0} \frac{1}{i!}\\
       T^{avg}(n) &\leq e
    \end{aligned}
\end{align*}
And since e is a constant it follows that:
\[ T^{avg}(n) \in O(1)\]

\end{adjustwidth} 




\end{document}