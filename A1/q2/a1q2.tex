% <- percent signs are used to comment
\documentclass[12pt]{article}

%%%%%% PACKAGES - this part loads additional material for LaTeX %%%%%%%%%
% Nearly anything you want can be done in LaTeX if you load the right package 
% (search ctan.org or google it if you are looking for something).  We will load
% here a few that we need for this document or that we expect you to need later.

% The next 3 lines are needed to fix shortcomings of TeX that only make sense given its 40-year history ...
% Simple keep and ignore.
\usepackage[utf8]{inputenc}
\usepackage[T1]{fontenc}
\usepackage{lmodern}
\usepackage{amsmath}
\usepackage{changepage}
\usepackage{lipsum}
\usepackage{bm}

% Custom margins (and paper sizes etc.) because LaTeX else wastes much space
\usepackage[margin=1in]{geometry}

% The following packages are created by the American Mathematical Society (AMS)
% and provide lots of tools for special fonts, symbols, theorems, and proof
\usepackage{amsmath,amsfonts,amssymb,amsthm}
% mathtools contains many detail improvements over ams and core tex
\usepackage{mathtools}

% graphicx is required for images
\usepackage{graphicx}

% enumitem used for customizing enumerations
\usepackage[shortlabels]{enumitem}

% tikz is the package used for drawing, in particular for drawing trees. You may also find simplified packages like tikz-qtree and forest useful
\usepackage{tikz}

% hyperref allows links, urls, and many other PDF tricks.  We load it here
%          in such a way that the PDF file has info about it
\usepackage[%
	pdftitle={CS240 Assignment 0},%
	hidelinks,%
]{hyperref}


%%%%%% COMMANDS - here you can define your own LaTeX-commands %%%%%%%%%

%%%%%% End of Preamble %%%%%%%%%%%%%

\begin{document}

\begin{center}
{\Large\textbf{CS240, Spring 2022}}\\
\vspace{2mm}
{\Large\textbf{Assignment 1: Question 2}}\\
\vspace{3mm}
\end{center}
\[ \]
\textbf{Q2a)} Determine the relationship between $f(n) = n^2 + 22n(\log(n)) + 13$ and $g(n) = n^2\log(n) + 14$:\\
\begin{adjustwidth}{0em}{0pt}
In order to find the relationship between $f(n)$ and $g(n)$ we will use limit $L$ such that:
\[ L = \lim_{n\to\infty}\frac{f(n)}{g(n)} \]
Replacing $f(n)$ and $g(n)$ with the polynomials they represent gives us:
\[ L = \lim_{n\to\infty}\left(\frac{n^2 + 22n(\log(n)) + 13}{n^2\log(n) + 14}\right) \]
The Comparison Test tells us that since $n^2\log(n) < n^2\log(n) + 14$ for all $n > 0$ that:
\[\text{if  } \lim_{n\to\infty}\left(\frac{n^2 + 22n(\log(n)) + 13}{n^2\log(n)}\right) = 0 \implies \lim_{n\to\infty}\left(\frac{n^2 + 22n(\log(n)) + 13}{n^2\log(n) + 14}\right) = 0 \]
Therefore we can simplify the left most equation to get:
\begin{align*}
    \begin{aligned}
       \lim_{n\to\infty}\left(\frac{n^2 + 22n(\log(n)) + 13}{n^2\log(n)}\right) &=\lim_{n\to\infty}\left(\frac{n^2}{n^2\log(n)}+\frac{22n(\log(n))}{n^2\log(n)}+\frac{13}{n^2\log(n)}\right)  \\
    	   &=\lim_{n\to\infty}\left(\frac{1}{\log(n)}+\frac{22}{n}+\frac{13}{n^2\log(n)}\right) \\
    	       	   &=\frac{1}{\log(\infty)}+\frac{22}{\infty}+\frac{13}{\infty^2\log(\infty)} \\
    	       	   &=0 + 0  + 0 \\
    	       	   &=0\\
    \end{aligned}
\end{align*}
Thus by the Comparison test we get:
\[ \lim_{n\to\infty}\left(\frac{n^2 + 22n(\log(n)) + 13}{n^2\log(n) + 14}\right) = L = 0 \]
Since $L = 0$ this means that:
\[ n^2 + 22n(log(n)) + 13 \in \textbf{o}(n^2log(n) + 14) \] 
\end{adjustwidth}
\newpage
\begin{adjustwidth}{0em}{0pt}
\textbf{Q2b)} Determine the relationship between $f(n) = \sqrt{n}$ and $g(n) = \log(n^4)$:\\ \\
We will find the relationship between $f(n)$ and $g(n)$ by finding the limit L where:
\[ L =  \lim_{n\to\infty}\frac{\sqrt{n}}{\log(n^4)} \]
Log rules show us that $\log(n^4)$ = $4\log(n)$ therefore our limit formula becomes:
\[ L =  \lim_{n\to\infty}\frac{\sqrt{n}}{4\log(n)} \]
If we apply L'Hopital's rule we get:
\begin{align*}
    \begin{aligned}
       \lim_{n\to\infty}\frac{\sqrt{n}}{4\log(n)} &= \lim_{n\to\infty}\frac{\frac{1}{2}n^{-\frac{1}{2}}}{\frac{4}{n\ln(2)}} \\
       &= \lim_{n\to\infty}\frac{\frac{1}{2}\times\frac{1}{n^{\frac{1}{2}}}}{\frac{4}{n\ln(2)}} \\
       &= \lim_{n\to\infty}\frac{\frac{1}{2}\times\frac{1}{n^{\frac{1}{2}}}\times{n}}{\frac{4}{\ln(2)}} \\
       &= \lim_{n\to\infty}\frac{\frac{1}{2}\sqrt{n}}{\frac{4}{\ln(2)}} \\
       &= \frac{\frac{1}{2}\sqrt{\infty}}{\frac{4}{\ln(2)}} \\
       &= \infty \\
    \end{aligned}
\end{align*}
Therefore by the limit test since $L = \infty$ it implies that the relationship is:
\[ \sqrt{n} \in \bm{\omega}(log(n^4)) \]\\
\end{adjustwidth}
\newpage
\begin{adjustwidth}{0em}{0pt}
\textbf{Q2c)} Determine the relationship between $f(n) = 10^n+99n^{10}$ and $g(n) = 75^n + 25n^{27}$:\\ \\
We will find the relationship between $f(n)$ and $g(n)$ by finding the limit L where:
\[ L =  \lim_{n\to\infty}\frac{10^n+99n^{10}}{75^n + 25n^{27}} \]
We will apply L'Hopital's rule to see if we notice any trends in the derivative:
\[ \lim_{n\to\infty}\frac{10^n+99n^{10}}{75^n + 12n^{27}} = \lim_{n\to\infty}\frac{10^n\times\ln(10)+990n^{9}}{75^n\times\ln(75) + 650n^{26}} \]
As we can see each time we take the derivative of $75^n$ or $10^n$ we will get a constant factor ($\ln(75)$ or $\ln(10)$) which wont impact future derivatives. Since the derivative will continue much the same we will skip to the 10th derivative:
\[ \lim_{n\to\infty}\frac{10^n+99n^{10}}{75^n + 12n^{27}} = \lim_{n\to\infty}\frac{10^n\times(\ln(10))^{10}+359251200}{75^n\times(\ln(75))^{10} + 25\times \frac{27!}{17!}n^{17}} \]
If we take the next derivative we will get rid of the constant term (359251200) as seen below:
\[ \lim_{n\to\infty}\frac{10^n+99n^{10}}{75^n + 12n^{27}} = \lim_{n\to\infty}\frac{10^n\times(\ln(10))^{11}}{75^n\times(\ln(75))^{11} + 25\times \frac{27!}{16!}n^{16}} \]
To get rid of the $\frac{27!}{16!}n^{16}$ term on the denominator we will skip to the 28th derivative:
\begin{align*}
    \begin{aligned}
       \lim_{n\to\infty}\frac{10^n+99n^{10}}{75^n + 12n^{27}} &= \lim_{n\to\infty}\frac{10^n\times(\ln(10))^{28}}{75^n\times(\ln(75))^{28}} \\
       &= \lim_{n\to\infty}(\frac{10}{75})^{n}\times(\frac{\ln(10)}{\ln(75)})^{25} \\
       &= (\frac{10}{75})^{\infty}\times(\frac{\ln(10)}{\ln(75)})^{25} \\
       &= 0\times(\frac{\ln(10)}{\ln(75)})^{25} \\
       &= 0
    \end{aligned}
\end{align*}
Therefore by the limit test since $L = 0$ it implies that the relationship is:
\[ 10^n+99n^{10} \in \textbf{o}(75^n + 12n^{27}) \]\\
\end{adjustwidth}

\end{document}