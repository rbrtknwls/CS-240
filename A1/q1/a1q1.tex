% <- percent signs are used to comment
\documentclass[12pt]{article}

%%%%%% PACKAGES - this part loads additional material for LaTeX %%%%%%%%%
% Nearly anything you want can be done in LaTeX if you load the right package 
% (search ctan.org or google it if you are looking for something).  We will load
% here a few that we need for this document or that we expect you to need later.

% The next 3 lines are needed to fix shortcomings of TeX that only make sense given its 40-year history ...
% Simple keep and ignore.
\usepackage[utf8]{inputenc}
\usepackage[T1]{fontenc}
\usepackage{lmodern}
\usepackage{amsmath}
\usepackage{changepage}
\usepackage{lipsum}

% Custom margins (and paper sizes etc.) because LaTeX else wastes much space
\usepackage[margin=1in]{geometry}

% The following packages are created by the American Mathematical Society (AMS)
% and provide lots of tools for special fonts, symbols, theorems, and proof
\usepackage{amsmath,amsfonts,amssymb,amsthm}
% mathtools contains many detail improvements over ams and core tex
\usepackage{mathtools}

% graphicx is required for images
\usepackage{graphicx}

% enumitem used for customizing enumerations
\usepackage[shortlabels]{enumitem}

% tikz is the package used for drawing, in particular for drawing trees. You may also find simplified packages like tikz-qtree and forest useful
\usepackage{tikz}

% hyperref allows links, urls, and many other PDF tricks.  We load it here
%          in such a way that the PDF file has info about it
\usepackage[%
	pdftitle={CS240 Assignment 0},%
	hidelinks,%
]{hyperref}


%%%%%% COMMANDS - here you can define your own LaTeX-commands %%%%%%%%%

%%%%%% End of Preamble %%%%%%%%%%%%%

\begin{document}

\begin{center}
{\Large\textbf{CS240, Spring 2022}}\\
\vspace{2mm}
{\Large\textbf{Assignment 1: Question 1}}\\
\vspace{3mm}
\end{center}
\[ \]
\textbf{Q1a)} Prove that $7n^4-5n^2+6 \in O(n^4)$\\
\begin{adjustwidth}{0em}{0pt}
We will start with the following observations:
\begin{align*}
    \begin{aligned}
       1)& \ n \leq n^4 \textrm{  which implies   } n \leq 6n^4 \ \ \ \ \ \ \forall n \geq 1\\
    	   2)& \ -n \leq 0 \textrm{  which implies   } -5n^2 \leq 0 \ \ \ \  \ \forall n \geq 1\\
    \end{aligned}
\end{align*}
Applying both identities to the equation we started with gives us: 
\[ 7n^4-5n^2+6 \leq 7n^4+6n^4  \ \ \ \ \ \ \forall n \geq 1 \]
Simplifying we get:
\[ 7n^4-5n^2+6 \leq 13n^4  \ \ \ \ \ \ \forall n \geq 1 \]
Thus {\boldmath$c = 13$} and {\boldmath$n_0 = 1$}, which by first principles proves that  $7n^4-5n^2+6 \in O(n^4)$.
\end{adjustwidth} 
\[ \]
\[ \]
\textbf{Q1b)} Prove that $7n^4-5n^2+6 \in \Omega(n^4)$\\
\begin{adjustwidth}{0pt}{0pt}
Again we will start with some more observations:
\begin{align*}
    \begin{aligned}
       1)& \ 7n^4 = 2n^4 + 5n^4 \\
    	   2)& \ 6 \geq 0 \\
    \end{aligned}
\end{align*}
Applying both identities to the equation we started with gives us: 
\[ 7n^4 - 5n^2 + 6 \geq 2n^4 + 5n^4 -5n^2 \]
We will now isolate the terms $5n^4 -5n^2$ and discover when the equation is positive:
\begin{align*}
    \begin{aligned}
       5n^4 -5n^2 &\geq 0 \\
    	   5n^4 &\geq 5n^2 \\
    	   n^2 &\geq 1 \\
    \end{aligned}
\end{align*}
Therefore when $n \geq 1$ the overall equation would be smaller if the terms are removed, so it follows that:
\[ 7n^4 - 5n^2 + 6 \geq 2n^4 + 5n^4 -5n^2 \geq 2n^4 \ \ \ \ \ \ \forall n \geq 1 \]
Thus {\boldmath$c = 2$} and {\boldmath$n_0 = 1$}, which by first principles proves that  $7n^4-5n^2+6 \in \Omega(n^4)$.
\end{adjustwidth}
\newpage
\textbf{Q1c)} Prove that $5n^2 + 15 \in o(n^3)$\\
\begin{adjustwidth}{0pt}{0pt}
We can start by making the following observation:
\begin{align*}
    \begin{aligned}
		1)& \ 1 \leq 1*n^2 \textrm{  which implies   } 15 \leq 15n^2 \ \ \ \ \ \ \forall n \geq 1\\
    \end{aligned}
\end{align*}
Therefore we can get the following equality:
\begin{align*}
    \begin{aligned}
       5n^2 + 15 &\leq 5n^2 + 15n^2 \ \ \ \ \ \ \forall n \geq 1 \\
    	   5n^2 + 15  &\leq 20n^2 \ \ \ \ \ \ \ \ \ \ \ \ \ \ \forall n \geq 1
    \end{aligned}
\end{align*}
In order to get a $n^3$ on the right side we can do the following:
\[ 5n^2 + 15 \leq \frac{20}{n}n^3 \ \ \ \ \ \ \forall n \geq 1 \]
For $5n^2 + 15 \in o(n^3)$, it must be the case that for any value of $c > 0$:
\[ 5n^2 + 15 \leq \frac{20}{n}n^3 \leq cn^3\ \ \ \ \ \ \forall n \geq 1 \]
At the moment $5n^2 + 15$ is not relevant so we can remove it and solve for $n_0$:
\begin{align*}
    \begin{aligned}
       \frac{20}{n}n^3 &\leq cn^3 \ \ \ \ \ \ \forall n \geq 1 \\
    	   n &\geq \frac{20}{c} \ \ \ \forall n \geq 1
    \end{aligned}
\end{align*}
Therefore $n \geq \frac{20}{c}$ and $n \geq 1$ which tells us that:
{\boldmath \[ n_0 = \lceil \frac{20}{c} \rceil \]}
Thus for any value $c > 0$ we can get a value for $n_0$ such that:
\[ 5n^2 + 15 \leq \frac{20}{n}n^3 \leq cn^3\ \ \ \ \ \ \forall n \geq n_0 \]
Which by first principles proves that $5n^2 + 15 \in o(n^3)$.
\end{adjustwidth}

\end{document}