% <- percent signs are used to comment
\documentclass[12pt]{article}

%%%%%% PACKAGES - this part loads additional material for LaTeX %%%%%%%%%
% Nearly anything you want can be done in LaTeX if you load the right package 
% (search ctan.org or google it if you are looking for something).  We will load
% here a few that we need for this document or that we expect you to need later.

% The next 3 lines are needed to fix shortcomings of TeX that only make sense given its 40-year history ...
% Simple keep and ignore.
\usepackage[utf8]{inputenc}
\usepackage[T1]{fontenc}
\usepackage{lmodern}
\usepackage{amsmath}
\usepackage{changepage}
\usepackage{lipsum}
\usepackage{bm}
\usepackage{ulem}


% Custom margins (and paper sizes etc.) because LaTeX else wastes much space
\usepackage[margin=1in]{geometry}

% The following packages are created by the American Mathematical Society (AMS)
% and provide lots of tools for special fonts, symbols, theorems, and proof
\usepackage{amsmath,amsfonts,amssymb,amsthm}
% mathtools contains many detail improvements over ams and core tex
\usepackage{mathtools}

% graphicx is required for images
\usepackage{graphicx}

% enumitem used for customizing enumerations
\usepackage[shortlabels]{enumitem}

% tikz is the package used for drawing, in particular for drawing trees. You may also find simplified packages like tikz-qtree and forest useful
\usepackage{tikz}

% hyperref allows links, urls, and many other PDF tricks.  We load it here
%          in such a way that the PDF file has info about it
\usepackage[%
	pdftitle={CS240 Assignment 0},%
	hidelinks,%
]{hyperref}


%%%%%% COMMANDS - here you can define your own LaTeX-commands %%%%%%%%%

%%%%%% End of Preamble %%%%%%%%%%%%%

\begin{document}

\begin{center}
{\Large\textbf{CS240, Spring 2022}}\\
\vspace{2mm}
{\Large\textbf{Assignment 1: Question 3}}\\
\vspace{3mm}
\end{center}
\[ \]
\textbf{Q3a)} Prove or disprove "If $f(n) \in O(g(n))$ then $g(n) \in \Omega(f(n))$":\\ 
\begin{adjustwidth}{0em}{0pt}
In order to prove this implication we will start by assuming the hypothesis, formally this tells us that there exists constants $c>0$ and $n_0>0$ such that:
\[ |f(n)| \leq c|g(n)| \ \ \ \forall n \geq n_0 \]
Dividing both sides by our constant $c$ gives us:
\[ \frac{1}{c}|f(n)| \leq |g(n)| \ \ \ \forall n \geq n_0 \]
Since c can be any real number (and contains all the values between 1 and 0), this implies that $c$ and $\frac{1}{c}$ will have the same range of values. In other words we can make a new variable $d = \frac{1}{c}$ and it will be the case that $d >0$. Therefore our equation becomes:
\[ d|f(n)| \leq |g(n)| \ \ \ \forall n \geq n_0 \]
Therefore by first principles it must be the case that (which proves the statement):
\[ g(n) \in \bm{\Omega}(f(n)) \] \\
\end{adjustwidth}

\begin{adjustwidth}{0em}{0pt}
\textbf{Q3b)} Prove or disprove "There exists $f(n), g(n)$ such that $f(n)\in o(g(n))$ and $f(n) \in \omega(g(n))$":\\ \\
Assuming the statement is correct formally tells us that $\forall c>0$, there exists an $n_0>0$ such that the two equations below are satisfied:
\begin{align*}
    \begin{aligned}
       1)& \ |f(n)| \leq c|g(n)| \ \ \ \forall n \geq n_0 \\
       2)& \ |f(n)| \geq c|g(n)| \ \ \ \forall n \geq n_0
    \end{aligned}
\end{align*}
Combining the two equation gives us:
\[ |f(n)| \leq c|g(n)| \leq |f(n)| \ \ \ \forall n \geq n_0 \\ \]
For each value of $|f(n)|$, $c$ will be every possible value. Therefore in order for $c|g(n)|$ to be bounded by $|f(n)|$, it must be the case that $|g(n)| = 0$. Therefore we get:
\[ |f(n)| \leq 0 \leq |f(n)| \ \ \ \forall n \geq n_0 \\ \]
This tells us that the only possible value for $|f(n)| = 0$. However since we are told $f(x),g(x) > 0$, it implies that our only solution for $f(x)$ and $g(x)$ is illegal and so:
\[ \text{ There exists } \textbf{no } f(n), g(n) \text{ such that } f(n)\in o(g(n)) \text{ and } f(n) \in \omega(g(n)) \]

\end{adjustwidth}

\newpage
\begin{adjustwidth}{0em}{0pt}
\textbf{Q3c)} Prove or disprove "If $f(n) \in O(g(n))$ then $2^{f(n)} \in O(2^{g(n)})$":\\ \\
We will start by considering the case where $f(n) = 2n^2$ and $g(n) = n^2$, in class we found these functions had the following relationship:
\[ 2n^2 \in O(n^2) \]
Assuming the implication is correct, this implies that if we take both functions to the power of 2 we should get:
\[ 2^{2n^2} \in O(2^{n^2}) \]
By first principles this means that there exists a $c > 0$ and $n_0 > 0$ such that:
\[ 2^{2n^2} \leq c2^{n^2}  \ \ \ \forall n \geq n_0 \\ \]
If we simplify we get:
\begin{align*}
    \begin{aligned}
       2^{n^2}\times2^{n^2} &\leq c2^{n^2}  \ \ \ \forall n \geq n_0 \\
       2^{n^2} &\leq c  \ \ \ \forall n \geq n_0 \\
       n^2 &\leq \log(c)  \ \ \ \forall n \geq n_0 \\
       n &\leq \sqrt{\log c}  \ \ \ \forall n \geq n_0 
    \end{aligned}
\end{align*}
This tells us that for any $n > \sqrt{\log c}$, that $2^{2n^2}$ grows faster then $2^{n^2}$. This proves the \textbf{statement is false}, as for all $n_0 = \sqrt{\log c}$ where $c >0$:
\[  2^{2n^2} \geq c2^{n^2}  \ \ \ \forall n \geq n_0 \]
This implies that:
\[ 2^{2n^2} \in \omega(2^{n^2}) \text{  and   } 2^{2n^2} \xout{\in} O(2^{n^2}) \]
And so we have \textbf{disproved} the statement by giving a counter example.
\end{adjustwidth}
\newpage
\begin{adjustwidth}{0em}{0pt}
\textbf{Q3d)} Prove or disprove "$(\log(n))^{\log(n)} \in O(n^2)$":\\ \\
We will start by evaluating the following function:
\[ \log(n)^{\log(n)} \]
In order to simplify we will set the following variable such that:
\[ a = \log(n) \text { and } n = 2^a \]
Plugging this into the original equation gives us:
\[ \log(n)^{\log(n)} = a^a \] 
We can also make the observation that:
\[ a^a \geq 4^a  \ \ \forall a \geq 4 \] 
Thus we can get the following inequality: 
\begin{align*}
    \begin{aligned}
       a^a &\geq 4^a  \\
       \log(n)^{\log(n)} &\geq 4^a \ \ \forall a \geq 4  \\
       \log(n)^{\log(n)} &\geq 4^{log(n)} \ \ \forall a \geq 4 \\
       \log(n)^{\log(n)} &\geq 2^{2log(n)} \ \ \forall a \geq 4 \\
       \log(n)^{\log(n)} &\geq 2^{log(n^2)} \ \ \forall a \geq 4 \\
       \log(n)^{\log(n)} &\geq n^2 \ \ \forall n \geq 16 \\
    \end{aligned}
\end{align*}
Therefore we have found $n_0 = 16$ and $c=1$ which by first principles tell us that: 
\[  \log(n)^{\log(n)} \xout{\in} O(n^2) \]
And so we have \textbf{disproved} the statement.
\end{adjustwidth}

\newpage
\begin{adjustwidth}{0em}{0pt}
\textbf{Q3e)} Prove or disprove "$\log(n)\times2^{\sin(n^3)} \in O(n)$":\\ \\
To start we can make the following observation:
\[ \sin(n^3) \leq 1 \] 
Therefore we can make the following inequality: 
\begin{align*}
    \begin{aligned}
       \log(n)\times2^{\sin(n^3)} &\leq \log(n)\times2^{1} \\
        &\leq 2\log(n)  \\
        &\leq 2n  \ (\text{ since } \log(n) \in O(n)) \forall n \geq 1 \\
    \end{aligned}
\end{align*}
Thus we have $c = 2$ and $n_0 = 1$ and so by first principles we get:
\[ \log(n)\times2^{\sin(n^3)} \in \bm{O}(n) \] 


\end{adjustwidth}


\end{document}