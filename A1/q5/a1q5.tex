% <- percent signs are used to comment
\documentclass[12pt]{article}

%%%%%% PACKAGES - this part loads additional material for LaTeX %%%%%%%%%
% Nearly anything you want can be done in LaTeX if you load the right package 
% (search ctan.org or google it if you are looking for something).  We will load
% here a few that we need for this document or that we expect you to need later.

% The next 3 lines are needed to fix shortcomings of TeX that only make sense given its 40-year history ...
% Simple keep and ignore.
\usepackage[utf8]{inputenc}
\usepackage[T1]{fontenc}
\usepackage{lmodern}
\usepackage{amsmath}
\usepackage{changepage}
\usepackage{lipsum}
\usepackage{bm}
\usepackage{ulem}


% Custom margins (and paper sizes etc.) because LaTeX else wastes much space
\usepackage[margin=1in]{geometry}

% The following packages are created by the American Mathematical Society (AMS)
% and provide lots of tools for special fonts, symbols, theorems, and proof
\usepackage{amsmath,amsfonts,amssymb,amsthm}
% mathtools contains many detail improvements over ams and core tex
\usepackage{mathtools}

% graphicx is required for images
\usepackage{graphicx}

% enumitem used for customizing enumerations
\usepackage[shortlabels]{enumitem}

% tikz is the package used for drawing, in particular for drawing trees. You may also find simplified packages like tikz-qtree and forest useful
\usepackage{tikz}

% hyperref allows links, urls, and many other PDF tricks.  We load it here
%          in such a way that the PDF file has info about it
\usepackage[%
	pdftitle={CS240 Assignment 0},%
	hidelinks,%
]{hyperref}


%%%%%% COMMANDS - here you can define your own LaTeX-commands %%%%%%%%%

%%%%%% End of Preamble %%%%%%%%%%%%%

\begin{document}

\begin{center}
{\Large\textbf{CS240, Spring 2022}}\\
\vspace{2mm}
{\Large\textbf{Assignment 1: Question 5}}\\
\vspace{3mm}
\end{center}
\[ \]
\textbf{Q5a)} Analyse the following pieces of pseudocode, and give a tight bound of the running time. \\
\begin{adjustwidth}{0em}{0pt}
To begin, lets look at the time complexity of the content the inner most loop (Line 4). We are doing a linear amount of work, which we will represent by the constant $c$. Thus:
\[ \text{ Inner loop conent's time complexity  } = c \]
The rest of the code can now be converted into a series of summations:
\begin{align*}
    \begin{aligned}
       &= \sum^{n}_{i=1}\sum^{i^2}_{j=1}\sum^{\log(n)}_{k=1} \text{(Inner loop conent)} \\
       &= \sum^{n}_{i=1}\sum^{i^2}_{j=1}\sum^{\log(n)}_{k=1}c
    \end{aligned}
\end{align*}
Using algebra we will simplify these summations:
\begin{align*}
    \begin{aligned}
        \sum^{n}_{i=1}\sum^{i^2}_{j=1}\sum^{\log(n)}_{k=1}c &= \sum^{n}_{i=1}\sum^{i^2}_{j=1}c\log(n) \\
       &= \sum^{n}_{i=1}c\log(n)(i^2) \\
       &= c\log(n) \sum^{n}_{i=1}(i^2) \\
    \end{aligned}
\end{align*}
Since $c\log(n)$ is not in terms of $i$ we can separate it out of the summation, therefore our equations becomes:
\begin{align*}
    \begin{aligned}
        c\log(n) \sum^{n}_{i=1}(i^2) &= c\log(n)\frac{n(n+1)(n+2)}{6} \\
       &= c\log(n)\frac{n(2n^2+3n+1)}{6} \\
       &= c\log(n)\frac{2n^3+3n^2+n}{6} \\
       &= \frac{2c}{6}n^3\log(n) + \frac{3c}{6}n^2\log(n) + \frac{c}{6}n\log(n)
    \end{aligned}
\end{align*}
Since we maintained equality we can conclude that the code has a time complexity of:
\[ \frac{2c}{6}n^3\log(n) + \frac{3c}{6}n^2\log(n) + \frac{c}{6}n\log(n) \in \bm{\Theta}(n^3\log(n)) \]
\end{adjustwidth}


\newpage
\begin{adjustwidth}{0em}{0pt}
\textbf{Q5b)} Analyse the following pieces of pseudocode, and give a tight bound of the running time. \\ \\
To start, we will again look at the content of inner most loop (line 5). Since we are doing a linear amount of work, which we will represent by the constant $c$. Thus:
\[ \text{ Inner loop conent } = c_1 \]
Therefore our first loop ($j = 1$ to $n$) will have a time complexity of:
\begin{align*}
    \begin{aligned}
       &= \sum^{n}_{j=1}c_1  \\
       &= c_1n
    \end{aligned}
\end{align*}
Each time the while loop iterates it will take $c_1n$ k, plus the constant amount of time to square $i$ and store it (line 6). We will represent this time taken to square $i$ by the constant $c_2$. Thus the while loop has an inner time complexity of:
\[ c_1n + c_2 \]
The while loop will run while $i < n$. We know that at the start of the while loop $i = 2$ and $i$ will square itself each iteration. Therefore:
\[ i = 2^{2^{\text{\# of iterations}}} \]
And so the loop will run while:
\begin{align*}
    \begin{aligned}
       2^{2^{\text{\# of iterations}}} &< n \\
       2^{\text{\# of iterations}} &< \log(n) \\
       {\text{\# of iterations}}&< \log(\log(n)) \\
    \end{aligned}
\end{align*}
Therefore the loop will terminate when $\text{\# of iterations} \geq log(n)$. We will now move forward using the "sloppy method" \\ \\
\textbf{Solving for upper bound:} The number of iterations before terminating will be upper bounded by $2\log(log(n))$. Thus our time complexity can be represented by:
\begin{align*}
    \begin{aligned}
       \text{Code's time complexity} &\leq \sum^{2\log(\log(n))}_{x=1}c_1n + c_2   \\
       &\leq (c_1n + c_2)2\log(\log(n)) \\
       &\leq 2c_1n\log(\log(n))+ 2c_2\log(\log(n))
    \end{aligned}
\end{align*}
And so our time complexity for the upper bound is:
\[ 2c_1n\log(\log(n))+ 2c_2\log(\log(n)) \in O(n\log(\log(n))) \]
\textbf{Solving for lower bound:} The number of iterations before terminating will be upper bounded by $\frac{\log(log(n))}{2}$. Thus our time complexity can be represented by:
\begin{align*}
    \begin{aligned}
       \text{Code's time complexity} &\geq \sum^{\frac{\log(log(n))}{2}}_{x=1}c_1n + c_2   \\
       &\geq (c_1n + c_2)\frac{\log(log(n))}{2} \\
       &\geq \frac{c_1n\log(log(n))}{2}+ \frac{c_2\log(log(n))}{2}
    \end{aligned}
\end{align*}
And so our time complexity for the lower bound is:
\[ \frac{c_1n\log(log(n))}{2}+ \frac{c_2\log(log(n))}{2} \in \Omega(n\log(\log(n))) \]
Therefore since the upper bound and lower bound are the same, we can conclude:
\[ \text{ Code's time complexity  } \in \bm{\Theta}(n\log(\log(n))) \]
\end{adjustwidth}

\newpage
\begin{adjustwidth}{0em}{0pt}
\textbf{Q5c)} Analyse the following pieces of pseudocode, and give a tight bound of the running time. \\ \\
To start, we will represent all lines where we do constant amounts of work (line 4, line 5-6, line 7) with the following constants $c_1, c_2, c_3$. Thus our time complexity for the content inside the inner while loop is:
\[ \text{ Inner while loop content } = c_2 \]
The time complexity for the content inside the outer while loop is:
\[ \text{ Outer while loop content } = c_1 + \text{ Inner while loop } + c_3 \]
Before we use the "sloppy" method, we will first try to represent the number of iterations for each while loop. Starting with the first while loop, we know that $j$ is equal to:
\[ j = n^3 - (\text{\# of inner loop iterations}) \]
Therefore the inner while loop will run while:
\begin{align*}
    \begin{aligned}
       j & > i \\
       n^3 - i(\text{\# of inner loop iterations}) &> i \\
       - i(\text{\# of inner loop iterations}) &> i - n^3 \\
       \text{\# of inner loop iterations} &< \frac{n^3}{i} - 1
    \end{aligned}
\end{align*}
Solving for the outer while loop, we know that we can express $i$ as:
\[ i = 1 + 5 (\text{\# of outer loop iterations}) \]
Therefore the outer loop will run while: 
\begin{align*}
    \begin{aligned}
       i &< 5n \\
       1 + 5 (\text{\# of outer loop iterations}) &< 5n \\
       5 (\text{\# of outer loop iterations}) &< 5n - 1 \\
       \text{\# of outer loop iterations} &< n - \frac{1}{5}
    \end{aligned}
\end{align*}
Therefore the inner while loop will terminate when $\text{(\# of inner loop iterations)} \geq n^3 - i$ and the outer while loop will terminate when $\text{(\# of outer loop iterations)} \geq n - \frac{1}{5}$. \\ \\
\textbf{Solving for upper bound:} The number of iterations before terminating the outer while loop will go through is upper bounded by $n+2$. Thus our time complexity can be represented by:
\[ \text{Code's time complexity} \leq \sum^{n+2}_{x=1} \left( c_1 + \sum^{n^3-i}_{y=1}c_2 + c_3 \right) \]
However since $x$ represents the number of outer loop iterations, we can express $i$ as:
\begin{align*}
    \begin{aligned}
       i &= 1 + 5 (\text{\# of outer loop iterations}) \\
       &= 1 + 5x
    \end{aligned}
\end{align*}
Solving for the time complexity we get the following:
\begin{align*}
    \begin{aligned}
       &\leq \sum^{n+2}_{x=1} \left( c_1 + \sum^{\frac{n^3}{i} - 1}_{y=1}c_2 + c_3 \right) \\
       &\leq \sum^{n+2}_{x=1} \left( c_1 + \frac{n^3}{i}c_2-c_2-5xc_2 + c_3 \right) \\
       &\leq \sum^{n+2}_{x=1} \left(\frac{n^3}{i}c_2 \right) + \sum^{n+2}_{x=1} \left(-5xc_2 \right)  + \sum^{n+2}_{x=1} \left( c_1 + -c_2 + c_3  \right) \\
       &\leq n^3c_2\sum^{n+2}_{x=1} \left(\frac{1}{1+5x}\right) + -5c_2\sum^{n+2}_{x=1} \left(x \right) + ( c_1 + -c_2 + c_3  )(n+2) \\
    \end{aligned}
\end{align*}
Note that $\sum^{n+2}_{x=1} \left(\frac{1}{1+5x}\right) \leq \sum^{n+2}_{x=1} \left(\frac{1}{x}\right)$ for all values of $n>0$ so we can substitute using the equation found it the textbook, we will let $c_4$ represent $\gamma + o(1)$  and get:
\begin{align*}
    \begin{aligned}
       &\leq  n^3c_2\sum^{n+2}_{x=1} \left(\frac{1}{x}\right) -5c_2\frac{(n+2)(n+3)}{2} + ( c_1 - c_2 + c_3  )(n+2) \\
       &\leq  n^3c_2(\log(n+2) + c_4) -5c_2\frac{n^2 + 5n + 6}{2} + ( c_1n -c_2n + c_3n ) + (2c_1 -2c_2+2c_3) \\
    \end{aligned}
\end{align*}
From this we can see that the upper bound on the time complexity is:
\[  n^3c_2(\log(n+2) + c_4) -5c_2\frac{n^2 + 5n + 6}{2} + ( c_1n -c_2n + c_3n ) + (2c_1 -2c_2+2c_3 \in O(n^3log(n)) \] \\ \\
\textbf{Solving for lower bound:} The number of iterations before terminating the outer while loop will go through is upper bounded by $n-2$. Thus our time complexity can be represented by:
\[ \text{Code's time complexity} \geq \sum^{n-2}_{x=1} \left( c_1 + \sum^{\frac{n^3}{i} - 1}_{y=1}c_2 + c_3 \right) \]
However since $x$ represents the number of outer loop iterations, we can express $i$ as:
\begin{align*}
    \begin{aligned}
       i &= 1 + 5 (\text{\# of outer loop iterations}) \\
       &= 1 + 5x
    \end{aligned}
\end{align*}
Solving for the time complexity we get the following:
\begin{align*}
    \begin{aligned}
       &\geq \sum^{n-2}_{x=1} \left( c_1 + \sum^{\frac{n^3}{i} - 1}_{y=1}c_2 + c_3 \right) \\
       &\geq \sum^{n-2}_{x=1} \left( c_1 + \frac{n^3}{i}c_2-c_2-5xc_2 + c_3 \right) \\
       &\geq \sum^{n-2}_{x=1} \left(\frac{n^3}{i}c_2 \right) + \sum^{n-2}_{x=1} \left(-5xc_2 \right)  + \sum^{n-2}_{x=1} \left( c_1 + -c_2 + c_3  \right) \\
       &\geq n^3c_2\sum^{n-2}_{x=1} \left(\frac{1}{1+5x}\right) + -5c_2\sum^{n-2}_{x=1} \left(x \right) + ( c_1 + -c_2 + c_3  )(n-2) \\
    \end{aligned}
\end{align*}
Note that $\sum^{n+2}_{x=1} \left(\frac{1}{1+5x}\right) \geq \sum^{n+2}_{x=1} \left(\frac{1}{20x}\right)$ for all values of $n>0$ so we can substitute using the equation found it the textbook, we will let $c_4$ represent $\gamma + o(1)$  and get:
\begin{align*}
    \begin{aligned}
       &\leq  n^3c_2\sum^{n+2}_{x=1} \left(\frac{1}{20x}\right) -5c_2\frac{(n+2)(n+3)}{2} + ( c_1 - c_2 + c_3  )(n-2) \\
       &\leq  n^3\frac{c_2}{20}(\log(n+2) + c_4) -5c_2\frac{n^2 + 5n + 6}{2} - ( c_1n -c_2n + c_3n ) + (2c_1 -2c_2+2c_3) \\
    \end{aligned}
\end{align*}
From this we can see that the lower bound on the time complexity is:
\[   n^3\frac{c_2}{20}(\log(n+2) + c_4) -5c_2\frac{n^2 + 5n + 6}{2} - ( c_1n -c_2n + c_3n ) + (2c_1 -2c_2+2c_3) \in \Omega(n^3\log(n)) \]
Therefore since the upper bound and lower bound are the same, we can conclude:
\[ \text{ Code's time complexity  } \in \bm{\Theta}(n^3\log(n)) \]
\end{adjustwidth}

\end{document}