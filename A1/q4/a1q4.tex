% <- percent signs are used to comment
\documentclass[12pt]{article}

%%%%%% PACKAGES - this part loads additional material for LaTeX %%%%%%%%%
% Nearly anything you want can be done in LaTeX if you load the right package 
% (search ctan.org or google it if you are looking for something).  We will load
% here a few that we need for this document or that we expect you to need later.

% The next 3 lines are needed to fix shortcomings of TeX that only make sense given its 40-year history ...
% Simple keep and ignore.
\usepackage[utf8]{inputenc}
\usepackage[T1]{fontenc}
\usepackage{lmodern}
\usepackage{amsmath}
\usepackage{changepage}
\usepackage{lipsum}
\usepackage{bm}
\usepackage{ulem}


% Custom margins (and paper sizes etc.) because LaTeX else wastes much space
\usepackage[margin=1in]{geometry}

% The following packages are created by the American Mathematical Society (AMS)
% and provide lots of tools for special fonts, symbols, theorems, and proof
\usepackage{amsmath,amsfonts,amssymb,amsthm}
% mathtools contains many detail improvements over ams and core tex
\usepackage{mathtools}

% graphicx is required for images
\usepackage{graphicx}

% enumitem used for customizing enumerations
\usepackage[shortlabels]{enumitem}

% tikz is the package used for drawing, in particular for drawing trees. You may also find simplified packages like tikz-qtree and forest useful
\usepackage{tikz}

% hyperref allows links, urls, and many other PDF tricks.  We load it here
%          in such a way that the PDF file has info about it
\usepackage[%
	pdftitle={CS240 Assignment 0},%
	hidelinks,%
]{hyperref}


%%%%%% COMMANDS - here you can define your own LaTeX-commands %%%%%%%%%

%%%%%% End of Preamble %%%%%%%%%%%%%

\begin{document}

\begin{center}
{\Large\textbf{CS240, Spring 2022}}\\
\vspace{2mm}
{\Large\textbf{Assignment 1: Question 4}}\\
\vspace{3mm}
\end{center}
\[ \]
\textbf{Q4a)} Prove that $f(n) \in O'(g(n))$ implies that $f(n) \in O(g(n))$:\\ 
\begin{adjustwidth}{0em}{0pt}
To prove the implication we will assume the hypothesis that $f(n) \in O'(g(n))$. Be definition this means that there exists a constant $c > 0$, that for all $n>0$ we have:
\[ f(n) \leq cg(n) \]
Since $f(n) > 0$ this means that $f(n) = |f(n)|$ and $g(n) = |g(n)|$ this implies:
\[ |f(n)| \leq c|g(n)| \]
By first principles this means that:
\[ f(n) \in O(g(n)) \]
\end{adjustwidth}

\begin{adjustwidth}{0em}{0pt}
\textbf{Q4b)} Prove that $f(n) \in O(g(n))$ implies that $f(n) \in O'(g(n))$:\\ \\
To prove the implication we will assume the hypothesis that $f(n) \in O(g(n))$. Be definition this means that there exists a constant $c > 0$, that for some $n_0>0$ we have:
\[ |f(n)| \leq c|g(n)| \ \ \ \ \ \forall n > n_0\]
Since $f(n), g(n) > 0$ we can remove the absolute value signs to get:
\[ f(n) \leq cg(n) \ \ \ \ \ \forall n > n_0\]
This tells us that if $n_0$ is bounded by $0 < n_0 < n$ we will get:
\[ f(n) > cg(n) \ \ \ \ \ \forall n > n_0\]
We will now analyse the fraction:
\[ \frac{g(n)}{cf(n)} \]
Since both $c>0$ and $g(n)>0$ this fraction will be continuous, moreover the range of this fraction is all the real numbers between $0$ and $n_0$. If we take the maximum value (called $x$) of the range it makes sense that:
\[ \frac{g(n)}{cf(n)} \leq x \]
If we rearrange we get: 
\[ g(n) \leq x\times{c}\times{f(n)} \]
Since x and c are constants this means that the equation is equvilient to the definition of O', which means that we have proved $g(n) \in O'(f(n))$ as wanted.
\end{adjustwidth}

\end{document}