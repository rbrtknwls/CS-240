% <- percent signs are used to comment
\documentclass[12pt]{article}

%%%%%% PACKAGES - this part loads additional material for LaTeX %%%%%%%%%
% Nearly anything you want can be done in LaTeX if you load the right package 
% (search ctan.org or google it if you are looking for something).  We will load
% here a few that we need for this document or that we expect you to need later.

% The next 3 lines are needed to fix shortcomings of TeX that only make sense given its 40-year history ...
% Simple keep and ignore.
\usepackage[utf8]{inputenc}
\usepackage[T1]{fontenc}
\usepackage{lmodern}
\usepackage{amsmath}
\usepackage{changepage}
\usepackage{lipsum}
\usepackage{caption}

% Custom margins (and paper sizes etc.) because LaTeX else wastes much space
\usepackage[margin=1in]{geometry}

% The following packages are created by the American Mathematical Society (AMS)
% and provide lots of tools for special fonts, symbols, theorems, and proof
\usepackage{amsmath,amsfonts,amssymb,amsthm}
% mathtools contains many detail improvements over ams and core tex
\usepackage{mathtools}

% graphicx is required for images
\usepackage{graphicx}

% enumitem used for customizing enumerations
\usepackage[shortlabels]{enumitem}

% tikz is the package used for drawing, in particular for drawing trees. You may also find simplified packages like tikz-qtree and forest useful
\usepackage{tikz}

% hyperref allows links, urls, and many other PDF tricks.  We load it here
%          in such a way that the PDF file has info about it
\usepackage[%
	pdftitle={CS240 Assignment 0},%
	hidelinks,%
]{hyperref}


%%%%%% COMMANDS - here you can define your own LaTeX-commands %%%%%%%%%

%%%%%% End of Preamble %%%%%%%%%%%%%

\begin{document}

\begin{center}
{\Large\textbf{CS240, Spring 2022}}\\
\vspace{2mm}
{\Large\textbf{Assignment 5: Question 5}}\\
\vspace{3mm}
\end{center}

\begin{adjustwidth}{0em}{0pt}
\textbf{Q5a)} Compute the failure array for the pattern $P=$\texttt{ississi}.
\begin{center}
	\begin{tabular}{|c|c|c|c|c|c|c|} \hline
		F[0] & F[1] & F[2] & F[3] & F[4] & F[5] & F[7]  \\ \hline
		0 & 0 & 0  & 1  & 2 & 3 & 4   \\ \hline
	\end{tabular}
\end{center}
\textbf{Q5b)} Show how to search for pattern in the text using the KMP algorithm. Indicate in a table such as Table ?? which characters of P were compared with which characters of T.
\begin{center}
	\begin{tabular}{|c|c|c|c|c|c|c|c|c|c|c|c|c|c|c|c|c|c|c|c|c|c|c|c|c|c|c|c|c|}
	\hline		m&i&s&s&i&s&s&a&u&g&a&m&i&s&m&i&s&s&i&s&s&i&p&p&i&m&i&l&l\\
	\hline
	\hline
	x&&&&&&&&&&&&&&&&&&&&&&&&&&&&\\
	\hline
	&i&s&s&i&s&s&x&&&&&&&&&&&&&&&&&&&&&\\
	\hline
	&&&&(i)&(s)&(s)&x&&&&&&&&&&&&&&&&&&&&&\\
	\hline
	&&&&&&&x&&&&&&&&&&&&&&&&&&&&&\\
	\hline
	&&&&&&&&x&&&&&&&&&&&&&&&&&&&&\\
	\hline
	&&&&&&&&&x&&&&&&&&&&&&&&&&&&&\\
	\hline
	&&&&&&&&&&x&&&&&&&&&&&&&&&&&&\\
	\hline
	&&&&&&&&&&&x&&&&&&&&&&&&&&&&&\\
	\hline
	&&&&&&&&&&&&i&s&x&&&&&&&&&&&&&&\\
	\hline
	&&&&&&&&&&&&&&x&&&&&&&&&&&&&&\\
	\hline
	&&&&&&&&&&&&&&&i&s&s&i&s&s&i&&&&&&&\\
	\hline
	\end{tabular}
\end{center}
\end{adjustwidth} 
\newpage

\end{document}