% <- percent signs are used to comment
\documentclass[12pt]{article}

%%%%%% PACKAGES - this part loads additional material for LaTeX %%%%%%%%%
% Nearly anything you want can be done in LaTeX if you load the right package 
% (search ctan.org or google it if you are looking for something).  We will load
% here a few that we need for this document or that we expect you to need later.

% The next 3 lines are needed to fix shortcomings of TeX that only make sense given its 40-year history ...
% Simple keep and ignore.
\usepackage[utf8]{inputenc}
\usepackage[T1]{fontenc}
\usepackage{lmodern}
\usepackage{amsmath}
\usepackage{changepage}
\usepackage{lipsum}
\usepackage{caption}
\usepackage{algorithm}
\usepackage{algorithmic}

% Custom margins (and paper sizes etc.) because LaTeX else wastes much space
\usepackage[margin=1in]{geometry}

% The following packages are created by the American Mathematical Society (AMS)
% and provide lots of tools for special fonts, symbols, theorems, and proof
\usepackage{amsmath,amsfonts,amssymb,amsthm}
% mathtools contains many detail improvements over ams and core tex
\usepackage{mathtools}

% graphicx is required for images
\usepackage{graphicx}

% enumitem used for customizing enumerations
\usepackage[shortlabels]{enumitem}

% tikz is the package used for drawing, in particular for drawing trees. You may also find simplified packages like tikz-qtree and forest useful
\usepackage{tikz}

% hyperref allows links, urls, and many other PDF tricks.  We load it here
%          in such a way that the PDF file has info about it
\usepackage[%
	pdftitle={CS240 Assignment 0},%
	hidelinks,%
]{hyperref}


%%%%%% COMMANDS - here you can define your own LaTeX-commands %%%%%%%%%

%%%%%% End of Preamble %%%%%%%%%%%%%

\begin{document}

\begin{center}
{\Large\textbf{CS240, Spring 2022}}\\
\vspace{2mm}
{\Large\textbf{Assignment 5: Question 1}}\\
\vspace{3mm}
\end{center}

\begin{adjustwidth}{0em}{0pt}
\textbf{Q1)} Give an algorithm to find the point with the maximum x-value in a 2D kd-tree, and analyze
its complexity. For maximum credit, your algorithm must run in o(n) time, where n is the
number of points in the kd-tree. For simplicity, you may assume that n is a power of 4. If the
run-time T(n) of your algorithm satisfies a recurrence relation that has already been seen in
class, you can take for granted the corresponding growth rate for T(n), without giving the
proof. \\\\
Assume that we already have a KD tree called K which for each split (both in terms of x and y) somewhat equally separates values into two parts.\\\\
When our algorithm encounters a split in terms of x, we will always recurse on the right child of K. This is because we are looking for the largest x-value, which should always be larger then our current split. When we reach a split in terms of y we should recurse on both sides as the largest x value can have any y value. \\\\
We will assume that we build the KD tree by doing an x split first, and we will use a boolean to represent the type of split. Our algorithm is thus:

\begin{algorithm}
\caption{KD Tree Max X Search}
\begin{algorithmic} 
\STATE \textbf{findMax}(kDTree K, Bool xSplit) \{
\IF {K is leaf} 
\STATE return k.x
\ENDIF
\IF {xSplit == true} 
\STATE return findMax(k.rightChild, false)
\ENDIF
\STATE // if we reach this stage, it means we should do a Y split
\STATE return max(findMax(k.rightChild, true), findMax(k.leftChild, true))
\end{algorithmic}
\end{algorithm} Note that half the values will go under the x split, and the other half will go under the y split and will additionally be split twice, thus our relation is (simplifying using known relations):
\begin{align}
   T(n) &=\frac{1}{2}T(n/2) + \frac{1}{2}\times2T(n/4) + O(1) \\
     & =\frac{1}{2}\sqrt{n} + \frac{1}{2}\times2T(n/4) + O(1)\\
     & =\frac{1}{2}\sqrt{n} + \frac{1}{2}\log(n)
\end{align}
Since we know $\sqrt{n}$ dominates $\log(n)$ we will do our limit test using $\frac{\sqrt{n}}{n}$:
\begin{align}
   \text{limit} &=\lim_{n\rightarrow\infty}\frac{\sqrt{n}}{n} \\
     & =\lim_{n\rightarrow\infty}\frac{1}{\sqrt{n}}\\
     & =0 
\end{align}
And so by the limit test we have:
\[ \frac{1}{2}\sqrt{n} + \frac{1}{2}\log(n) = o(n)\]

\end{adjustwidth}
\end{document}