% <- percent signs are used to comment
\documentclass[12pt]{article}

%%%%%% PACKAGES - this part loads additional material for LaTeX %%%%%%%%%
% Nearly anything you want can be done in LaTeX if you load the right package 
% (search ctan.org or google it if you are looking for something).  We will load
% here a few that we need for this document or that we expect you to need later.

% The next 3 lines are needed to fix shortcomings of TeX that only make sense given its 40-year history ...
% Simple keep and ignore.
\usepackage[utf8]{inputenc}
\usepackage[T1]{fontenc}
\usepackage{lmodern}
\usepackage{amsmath}
\usepackage{changepage}
\usepackage{lipsum}

% Custom margins (and paper sizes etc.) because LaTeX else wastes much space
\usepackage[margin=1in]{geometry}

% The following packages are created by the American Mathematical Society (AMS)
% and provide lots of tools for special fonts, symbols, theorems, and proof
\usepackage{amsmath,amsfonts,amssymb,amsthm}
% mathtools contains many detail improvements over ams and core tex
\usepackage{mathtools}

% graphicx is required for images
\usepackage{graphicx}

% enumitem used for customizing enumerations
\usepackage[shortlabels]{enumitem}

% tikz is the package used for drawing, in particular for drawing trees. You may also find simplified packages like tikz-qtree and forest useful
\usepackage{tikz}

% hyperref allows links, urls, and many other PDF tricks.  We load it here
%          in such a way that the PDF file has info about it
\usepackage[%
	pdftitle={CS240 Assignment 0},%
	hidelinks,%
]{hyperref}


%%%%%% COMMANDS - here you can define your own LaTeX-commands %%%%%%%%%

%%%%%% End of Preamble %%%%%%%%%%%%%

\begin{document}

\begin{center}
{\Large\textbf{CS240, Spring 2022}}\\
\vspace{2mm}
{\Large\textbf{Assignment 2: Question 4}}\\
\vspace{3mm}
\end{center}
\[ \]
\textbf{Q4a)} On the example $(15,10,\dots)$ above, show us what these heaps would contain at each of the 5 steps (we don't know if these are min-heaps or
max-heaps yet, so just tell us what elements they contain).\\
\begin{adjustwidth}{0em}{0pt}
\begin{center}
\begin{tabular}{|c | c | c |} 
 \hline
 Steps & $H_{lo}$ & $H_{hi}$ \\ [0.5ex] 
 \hline\hline
 1 & 15 &  \\ 
 \hline
 2 & 10 & 15 \\
 \hline
 3 & 1, 10 & 15\\
 \hline
 4 & 1, 10 & 15, 20 \\
 \hline
 5 & 1, 10, 15 & 20, 30 \\ [1ex] 
 \hline
\end{tabular}
\end{center}

\end{adjustwidth} 

\begin{adjustwidth}{0em}{0pt}
\textbf{Q4b)} We would like to be able to read off the (current) median using just one access to $H_{\rm lo}$. What kind of heap should it be, a min-heap or a max-heap? How long does finding the current median take? \\ \\
We should use a max-heap, finding the current median should take $O(1)$ time as we are accessing the root of $H_{lo}$


\end{adjustwidth} 

\begin{adjustwidth}{0em}{0pt}
\textbf{Q4c)} Describe how to update the two heaps when inserting the next element. In particular, in which heap do you insert the element, and how do you ensure that $H_{\rm lo}$ and $H_{\rm hi}$ have the required size
afterwards? Give the runtime of your update method, with a short
justification; it should be $o(n)$. (At this stage, you will have to explain whether $H_{\rm hi}$ should be a min-heap or a max-heap.) \\ \\
We know that the maximum of $H_{lo}$ and the minimum of $H_{hi}$ is stored in the roots. We can check if the new element is greater then the max of $H_lo$ we can insert it into $H_{hi}$. If element is less then the minimum of $H_hi$ we can insert into $H_lo$ \\

If the counter for $H_{lo}$ is larger then $H_hi$+1 then we can use deleteMax for $H_{lo}$ after storing the value in a temporary value, and vise versa for if $H_{hi}$ is larger then $H_lo$+1. We will then apply a fix down to make sure the order remains the same\\

Note that our inserting, deleting (min/max) will take $O(log(n))$ and from our lectures we know that $log(n) \in o(n)$. Thus it follows that our algorthium will have a time complexity of $o(n)$ as required.
\end{adjustwidth} 




\end{document}