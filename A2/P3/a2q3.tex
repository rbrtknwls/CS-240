% <- percent signs are used to comment
\documentclass[12pt]{article}

%%%%%% PACKAGES - this part loads additional material for LaTeX %%%%%%%%%
% Nearly anything you want can be done in LaTeX if you load the right package 
% (search ctan.org or google it if you are looking for something).  We will load
% here a few that we need for this document or that we expect you to need later.

% The next 3 lines are needed to fix shortcomings of TeX that only make sense given its 40-year history ...
% Simple keep and ignore.
\usepackage[utf8]{inputenc}
\usepackage[T1]{fontenc}
\usepackage{lmodern}
\usepackage{amsmath}
\usepackage{changepage}
\usepackage{lipsum}

% Custom margins (and paper sizes etc.) because LaTeX else wastes much space
\usepackage[margin=1in]{geometry}

% The following packages are created by the American Mathematical Society (AMS)
% and provide lots of tools for special fonts, symbols, theorems, and proof
\usepackage{amsmath,amsfonts,amssymb,amsthm}
% mathtools contains many detail improvements over ams and core tex
\usepackage{mathtools}

% graphicx is required for images
\usepackage{graphicx}

% enumitem used for customizing enumerations
\usepackage[shortlabels]{enumitem}

% tikz is the package used for drawing, in particular for drawing trees. You may also find simplified packages like tikz-qtree and forest useful
\usepackage{tikz}

% hyperref allows links, urls, and many other PDF tricks.  We load it here
%          in such a way that the PDF file has info about it
\usepackage[%
	pdftitle={CS240 Assignment 0},%
	hidelinks,%
]{hyperref}


%%%%%% COMMANDS - here you can define your own LaTeX-commands %%%%%%%%%

%%%%%% End of Preamble %%%%%%%%%%%%%

\begin{document}

\begin{center}
{\Large\textbf{CS240, Spring 2022}}\\
\vspace{2mm}
{\Large\textbf{Assignment 2: Question 3}}\\
\vspace{3mm}
\end{center}
\[ \]
\textbf{Q3)} Given an array $A[0\ldots n-1]$ of numbers, show that if $A[i] \geq A[i-j]$ for all $j \geq \log(n)$, the array can be sorted in $O(n \log(\log(n)))$ time.

\smallskip
\noindent \textit{Hint:} Partition $A$ into contiguous blocks of size
$\log (n)$; i.e.  the first $\log (n)$ elements are in the first
block, the next $\log (n)$ elements are in the second block, and so
on. Then, establish a connection between the elements within two
blocks that are separated by another block.\\
\begin{adjustwidth}{0em}{0pt}

To start we will use the hint, we will partition the array into $n/log(n)$ blocks where each block has a size of $log(n)$. Since we are given that $A[i] \geq A[i-j]$ for all $j \geq \log(n)$, this means that each block's elements will be less then the elements to its right as they are a log(n) apart. \\ \\
Moving forward we will use merge sort on each of the block, since there are $n/log(n)$ blocks where each one will take $O(\log(n)\log(\log(n)))$ (as merge sort takes $O(n\log(n))$ and the size of each block is $\log(n)$). Therefore the total time complexity will be:
\[ O(\frac{n}{\log{n}}\log(n)\log(\log(n))) \]
Which simplifies to:
\[ O(n\log(\log(n))) \]

There are a total of $\frac{n}{\log{n}}$ blocks that we will need to merge, the merge itself is constant we need to multiply it by the number of blocks we need to do, which gives us a time complexity of $O(\frac{n}{\log{n}})$. Thus our total time complexity is:
\[ T(n) = O(n\log(\log(n))) + O(\frac{n}{\log{n}}) \]
The overall time complexity will be determined by the term that grows the fastest so we will use the lime test to compare them:
\begin{align*}
    \begin{aligned}
       L &= \lim_{n->\infty}\frac{n\log(\log(n))}{\frac{n}{\log{n}}}  \\
       &= \lim_{n->\infty}\frac{\log(\log(n))}{\frac{1}{\log{n}}} \\
       &= \lim_{n->\infty}\log(\log(n))\log(n) \\
       &= \infty 
    \end{aligned}
\end{align*}
Therefore since $\frac{n}{\log{n}} \in o(n\log(\log(n)))$, it means that $n\log(\log(n))$ grows faster. It follows that our total time complexity will become:
\[ T(n) = O(n\log(\log(n))) \]
Which is what we set up to prove.
\end{adjustwidth} 




\end{document}