% <- percent signs are used to comment
\documentclass[12pt]{article}

%%%%%% PACKAGES - this part loads additional material for LaTeX %%%%%%%%%
% Nearly anything you want can be done in LaTeX if you load the right package 
% (search ctan.org or google it if you are looking for something).  We will load
% here a few that we need for this document or that we expect you to need later.

% The next 3 lines are needed to fix shortcomings of TeX that only make sense given its 40-year history ...
% Simple keep and ignore.
\usepackage[utf8]{inputenc}
\usepackage[T1]{fontenc}
\usepackage{lmodern}
\usepackage{amsmath}
\usepackage{changepage}
\usepackage{lipsum}

% Custom margins (and paper sizes etc.) because LaTeX else wastes much space
\usepackage[margin=1in]{geometry}

% The following packages are created by the American Mathematical Society (AMS)
% and provide lots of tools for special fonts, symbols, theorems, and proof
\usepackage{amsmath,amsfonts,amssymb,amsthm}
% mathtools contains many detail improvements over ams and core tex
\usepackage{mathtools}

% graphicx is required for images
\usepackage{graphicx}

% enumitem used for customizing enumerations
\usepackage[shortlabels]{enumitem}

% tikz is the package used for drawing, in particular for drawing trees. You may also find simplified packages like tikz-qtree and forest useful
\usepackage{tikz}

% hyperref allows links, urls, and many other PDF tricks.  We load it here
%          in such a way that the PDF file has info about it
\usepackage[%
	pdftitle={CS240 Assignment 0},%
	hidelinks,%
]{hyperref}


%%%%%% COMMANDS - here you can define your own LaTeX-commands %%%%%%%%%

%%%%%% End of Preamble %%%%%%%%%%%%%

\begin{document}

\begin{center}
{\Large\textbf{CS240, Spring 2022}}\\
\vspace{2mm}
{\Large\textbf{Assignment 4: Question 1}}\\
\vspace{3mm}
\end{center}
\begin{adjustwidth}{0em}{0pt}
\textbf{Q1)} Specify all of the following in big-O in terms of f, n, and t (as necessary). What is the maximum height of T? What is the total space usage of T and all A(u)? What time is needed to search for a key k in T ? \\\\
Since each node contains a pointer to every element in our alphabet $\{0, 1, ..., 2^t-1\}$, this means we need to determine how many elements in the alphabet fit inside of $x_i$. In other words our maximum number of nodes needed to store $x_i$
\[ \text{number of nodes needed to store } x_i = 1 + \log_{2^t}(x_i)\]
Since we are given that:
\[ x_i <= n^f \]
This means that the maximum number of nodes we will ever need to store (aka the height of the tree is given by):
\begin{align*}
    \begin{aligned}
       \text{height} &<= \log_{2^t}(n^f) \\
       &<= f\times\log_{2^t}(n) \\
    \end{aligned}
\end{align*}
Using log rules we can simplify the log equation to get:
\begin{align*}
    \begin{aligned}
       \text{height} &<= 1 + f\times\log_{2^t}(n) \\
       &<= 1 + f\times\frac{\log_{2}(n)}{\log_{2}(2^t)} \\
       &<= 1+ f\times\frac{\log_{2}(n)}{t} \\
       &<= 1+ \frac{f}{t}\times\log_{2}(n) \\
       &\in O(\frac{f}{t}\times\log_{2}(n))
    \end{aligned}
\end{align*}
Conversationally in order to find a node within our tree, at most we would need to traverse through the height of the tree thus we get:
\[ \text{ Time to find a child } = O(\frac{f}{t}\times\log_{2}(n)) \]
Since each node has $2^t$ children this implies that the total number of elements that exist at level $i$ (where $i$ is an integer greater then or equal to 0) is given by:
\[ \text{number of elements at level i in the tri} = (2^t)^i\]
Therefore the total number of nodes is given by:
\begin{align*}
    \begin{aligned}
      \text{ \# of nodes } &= \sum_{i=0}^{1 + \frac{f}{t}\times\log_{2}(n)} (2^t)^i \\
       &= \frac{(2^t)^{\frac{f}{t}\times\log_{2}(n)} - 1}{2^t -1} \\
       &= \frac{(2^f)^{\log_{2}(n)} - 1}{2^t -1} \\
       &= \frac{n^f - 1}{2^t -1} \\
       &\in O(\frac{n^f}{2^t}) \\
    \end{aligned}
\end{align*}
Since each node contains an array of A(U), where the size of A(U) is $2^t+1$, our total space complexity is:
\[ \text{ space complexity} \in O(\frac{n^f}{2^t}\times(2^t+1)) \]
\[ \text{ space complexity} \in O(n^f) \]
\end{adjustwidth} 



\end{document}