% <- percent signs are used to comment
\documentclass[12pt]{article}

%%%%%% PACKAGES - this part loads additional material for LaTeX %%%%%%%%%
% Nearly anything you want can be done in LaTeX if you load the right package 
% (search ctan.org or google it if you are looking for something).  We will load
% here a few that we need for this document or that we expect you to need later.

% The next 3 lines are needed to fix shortcomings of TeX that only make sense given its 40-year history ...
% Simple keep and ignore.
\usepackage[utf8]{inputenc}
\usepackage[T1]{fontenc}
\usepackage{lmodern}
\usepackage{amsmath}
\usepackage{changepage}
\usepackage{lipsum}

% Custom margins (and paper sizes etc.) because LaTeX else wastes much space
\usepackage[margin=1in]{geometry}

% The following packages are created by the American Mathematical Society (AMS)
% and provide lots of tools for special fonts, symbols, theorems, and proof
\usepackage{amsmath,amsfonts,amssymb,amsthm}
% mathtools contains many detail improvements over ams and core tex
\usepackage{mathtools}

% graphicx is required for images
\usepackage{graphicx}

% enumitem used for customizing enumerations
\usepackage[shortlabels]{enumitem}

% tikz is the package used for drawing, in particular for drawing trees. You may also find simplified packages like tikz-qtree and forest useful
\usepackage{tikz}

% hyperref allows links, urls, and many other PDF tricks.  We load it here
%          in such a way that the PDF file has info about it
\usepackage[%
	pdftitle={CS240 Assignment 0},%
	hidelinks,%
]{hyperref}


%%%%%% COMMANDS - here you can define your own LaTeX-commands %%%%%%%%%

%%%%%% End of Preamble %%%%%%%%%%%%%

\begin{document}

\begin{center}
{\Large\textbf{CS240, Spring 2022}}\\
\vspace{2mm}
{\Large\textbf{Assignment 4: Question 4}}\\
\vspace{3mm}
\end{center}
\begin{adjustwidth}{0em}{0pt}
\textbf{Q4a)} Which hash function is better? Justify your answer.\\\\
To compare the two hash functions, lets consider their values for the first 7 elements of our universe:
\begin{center}
\begin{tabular}{||c | c c||}
	\hline
    k & Output of h1(k) & Output of h2(k) \\
    \hline\hline
    1 & 1 & 2 \\
    \hline
    2 & 2 & 4 \\
    \hline
    3 & 3 & 0 \\
    \hline
    4 & 4 & 2 \\
    \hline
    5 & 5 & 4 \\
    \hline
    6 & 0 & 0 \\
    \hline
    7 & 1 & 2 \\
    \hline
\end{tabular}
\end{center}
From this table we can see that the output of h1(k) encompasses the set \{0,1,2,3,4,5\} where as the output of h2(k) only encompasses \{2,4,0\}. \\\\
Therefore since the output of h2(k) only has half as many outputs as h1(k) we would expect more conflicts and thus h1(k) is the better hash function. \\\\\\
\textbf{Q4b)} Prove that $\lim\limits_{n\to \infty}\frac{c_n}{n} = \frac{1}{e}$.\\\\
To start we know that the number of keys and the number of spaces in our hash table are both equal to n. Moving forward we will define an indicator variable I such that (where i is an integer between 0 and n-1):
\[ I_i = \begin{cases}
0 \ \ \text{if the hash table at index i has more then 0 elements chained in it} \\
1 \ \ \text{if the has table at index i has 0 elements in it}
\end{cases} \]
Since $c_n$ represents the expected number of empty elements in the hash table, therefore we get:
\[ c_n = \sum^{n}_{i=0}I_i \]
An index will be empty if each of the n numbers that are inserted into the table go into any of the $n-1$ slots in the hash table. Therefore the expected value of any slot being empty is given by
\[ E[I_i] = (\frac{n-1}{n})^n \]
Plugging this into our equation for $c_n$ we get:
\begin{align*}
    \begin{aligned}
       c_n &= \sum^{n}_{i=0}(\frac{n-1}{n})^n  \\
       c_n &= n(\frac{n-1}{n})^n  \\
    \end{aligned}
\end{align*}
Since we are asked to solve for $\lim\limits_{n\to \infty}\frac{c_n}{n}$ we can now plug our equation for $c_n$ in to get:
\begin{align*}
    \begin{aligned}
       \lim\limits_{n\to \infty}\frac{c_n}{n} &= \lim\limits_{n\to \infty}\frac{n(\frac{n-1}{n})^n}{n}  \\
       &= \lim\limits_{n\to \infty}(\frac{n-1}{n})^n  \\
       &= \lim\limits_{n\to \infty}(1-\frac{1}{n})^n  \\
       &= \frac{1}{e} \ \ \ \text{from the fact given in the question} \\
    \end{aligned}
\end{align*}
Thus we have proved $\lim\limits_{n\to \infty}\frac{c_n}{n} = \frac{1}{e}$ as required.




\end{adjustwidth} 
\newpage





\end{document}